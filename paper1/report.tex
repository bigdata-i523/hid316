\documentclass[sigconf]{acmart}

\usepackage{hyperref}

\usepackage{endfloat}
\renewcommand{\efloatseparator}{\mbox{}} % no new page between figures

\usepackage{booktabs} % For formal tables

\settopmatter{printacmref=false} % Removes citation information below abstract
\renewcommand\footnotetextcopyrightpermission[1]{} % removes footnote with conference information in first column
\pagestyle{plain} % removes running headers

\begin{document}
\title{Big Data Analytics in Biometric Identity Management}


\author{Robert W. Gasiewicz}
\orcid{1234-5678-9012}
\affiliation{%
  \institution{Indiana University}
  \streetaddress{711 N. Park Avenue}
  \city{Bloomington} 
  \state{IN} 
  \postcode{47408}
}
\email{rgasiewi@iu.edu}

\begin{abstract}
An understanding how biometrics is changing rapidly, and with it, both the size and scope of data being collected.            From 2-print to 10-print, iris to facial recognition, the demand for both data intensive processes and rapid matching have grown exponentially, a case study in big data if there ever was one.
\end{abstract}

\keywords{i523}


\maketitle

\section{Introduction}

The \textit{proceedings} are the records of a
conference. ACM seeks to give these
conference by-products a uniform, high-quality appearance.  To do
this, ACM has some rigid requirements for the format of the
proceedings documents: there is a specified format (balanced double
columns), a specified set of fonts (Arial or Helvetica and Times
Roman) in certain specified sizes, a specified live area, centered on
the page, specified size of margins, specified column width and gutter
size \cite{editor00}.


\begin{acks}

  The authors would like to thank 

\end{acks}

\bibliographystyle{ACM-Reference-Format}
\bibliography{report} 

\end{document}
