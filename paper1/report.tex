\documentclass[sigconf]{acmart}

\usepackage{hyperref}

\usepackage{endfloat}
\renewcommand{\efloatseparator}{\mbox{}} % no new page between figures

\usepackage{booktabs} % For formal tables

\settopmatter{printacmref=false} % Removes citation information below abstract
\renewcommand\footnotetextcopyrightpermission[1]{} % removes footnote with conference information in first column
\pagestyle{plain} % removes running headers

\begin{document}
\title{Big Data Analytics in Biometric Identity Management}


\author{Robert W. Gasiewicz}
\orcid{1234-5678-9012}
\affiliation{%
  \institution{Indiana University}
  \streetaddress{711 N. Park Avenue}
  \city{Bloomington} 
  \state{IN} 
  \postcode{47408}
}
\email{rgasiewi@iu.edu}

\begin{abstract}
This paper is intended to be a primer for understanding how the United States Government, through its collection and use of biometric data, has leveraged big data in order to protect its citizens and keep our country safe. The speed and accuracy with which this biometric data can be effectively matched to an identity can mean the difference between life and death, as well as the integrity of our institutions. This paper predominantly focuses on how the Office of Biometric Identity Management, a division within the U.S. Department of Homeland Security, collects, stores, and uses this data to enhance national security. 
\end{abstract}

\keywords{i523, HID316, Big Data, Biometrics, Fingerprinting, 2-Print, 10-Print, Matchers, Matching Algorithms, DHS, Homeland Security, Border Security, National Security, Immigration, Terrorism, FBI, AFIS}


\maketitle

\section{Introduction}

Across the spectrum, big data is rapidly changing the way we do business, the way we live, and the way governments around the world do everything they can to keep us safe in the face of an increasingly dangerous world. Long before the advent of big data, fingerprints were used as a means of forensic identification, but it wasn't until technology had progressed to the point to which these prints could be converted and stored in digital format, organized, and then matched against other stored data and even other databases, that this data truly became useful on the large scale that it is today. Biometrics technology is changing rapidly, and with it, both the size and scope of data being collected. From 2-print to 10-print, iris to facial recognition, the demand for both data intensive processes and rapid matching have grown exponentially, and understanding how the United States Government uses biometrics is a case study in big data if there ever was one. 

\section{History of Fingerprinting: The Analog Era}

In 1858, a man by the name of Sir William James Herschel began using fingerprints as a means of identification \cite{editor00a} near Calcutta, India. This started as a means of not solving crimes, but preventing them; Sir William's aim was to thwart attempts at forging signatures - something that had begun to occur at epidemic proportions. Herschel also used fingerprinting to prevent the collection of pension benefits by relatives after the pensioner had deceased. It wasn't until 1886 that Scottish surgeon, Dr. Henry Faulds, proposed the concept of using fingerprints to identify criminals to London's Metropolitan Police \cite{editor00}. Incredibly, they dismissed his proposal. 

By 1906, the concept of identifying criminals using fingerprints had made its way to the United States, first in New York City and then elsewhere throughout the country. In 1924, the United States Congress created the Identification Division of the FBI and 22 years later, they had processed over 100 million fingerprint cards. By 1971, this number had more than doubled \cite{Editor00b}.

\section{Biometrics in the Digital Age}



\begin{acks}

  The authors would like to thank 

\end{acks}

\bibliographystyle{ACM-Reference-Format}
\bibliography{report} 

\end{document}
